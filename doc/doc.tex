\documentclass[10pt,a4paper]{article}
\usepackage[utf8]{inputenc}
\usepackage{hyperref}
\usepackage{listings}
\usepackage{color}
\definecolor{dkgreen}{rgb}{0,0.6,0}
\definecolor{gray}{rgb}{0.5,0.5,0.5}
\definecolor{mauve}{rgb}{0.58,0,0.82}

\lstset{frame=tb,
  language=sh,
  aboveskip=3mm,
  belowskip=3mm,
  showstringspaces=false,
  columns=flexible,
  basicstyle={\small\ttfamily},
  numbers=none,
  numberstyle=\tiny\color{gray},
  keywordstyle=\color{blue},
  commentstyle=\color{dkgreen},
  stringstyle=\color{mauve},
  breaklines=true,
  breakatwhitespace=true,
  tabsize=3
}

\author{Josué Alvarez}
\title{Simso web - Developper Documentation}
\begin{document}
\maketitle
\newpage
\section{Preamble}
This document is a non-exhaustive developer documentation, whose goal is to help future maintainers of the simso-web application to get started with the code. 
This document is written with the assumption that the developper is using an UNIX environment.

\section{Getting Started}
\paragraph{Introduction}
Simso web is a graphical interface built on top of simso. It runs as a full-client application (no server-side) written in javascript, and uses PypyJS (a javascript implementation of \href{"http://www.pypy.org/"}{Pypy}) to run Python in order to execute simso.
The main frameworks/tools used for the front-end development are \href{"https://angularjs.org/"}{Angular JS} and \href{"http://getbootstrap.com/"}{Bootstrap}.

\paragraph{Getting the code}
The code is available on github here : https://github.com/MaximeCheramy/simso-web. 
As simso-web embbeds its own version of simso, it is included in the form of a git submodule.

To setup your working copy, you have to run the following commands :
\begin{lstlisting}
git clone https://github.com/MaximeCheramy/simso-web.git
cd simso-web/submodules/simso
git submodule init
git submodule update
\end{lstlisting}

\paragraph{Architecture}



\end{document}